% Autogenerated translation of README.md by Texpad
% To stop this file being overwritten during the typeset process, please move or remove this header

\documentclass[12pt]{book}
\usepackage{graphicx}
\usepackage[utf8]{inputenc}
\usepackage[a4paper,left=.5in,right=.5in,top=.3in,bottom=0.3in]{geometry}
\setlength\parindent{0pt}
\setlength{\parskip}{\baselineskip}
\renewcommand*\familydefault{\sfdefault}
\usepackage{hyperref}
\pagestyle{plain}
\begin{document}
\Large

\chapter*{htmlize --- Convert buffer text and decorations to HTML}

\href{https://melpa.org/#/htmlize}{\includegraphics{https://melpa.org/packages/htmlize-badge.svg}}

This package converts the buffer text and the associated
decorations to HTML.  Mail to \href{mailto:hniksic@gmail.com}{hniksic@gmail.com} to discuss
features and additions.  All suggestions are more than welcome.

To use it, just switch to the buffer you want HTML-ized and type
\texttt{$<$kbd$>$}M-x htmlize-buffer\texttt{$<$/kbd$>$}.  You will be switched to a new buffer
that contains the resulting HTML code.  You can edit and inspect this
buffer, or you can just save it with \texttt{$<$kbd$>$}C-x C-w\texttt{$<$/kbd$>$}.  \texttt{$<$kbd$>$}M-x
htmlize-file\texttt{$<$/kbd$>$} will find a file, fontify it, and save the HTML
version in \texttt{FILE.html}, without any additional intervention.  \texttt{$<$kbd$>$}M-x
htmlize-many-files\texttt{$<$/kbd$>$} allows you to htmlize any number of files in
the same manner.  \texttt{$<$kbd$>$}M-x htmlize-many-files-dired\texttt{$<$/kbd$>$} does the
same for files marked in a dired buffer.

htmlize supports three types of HTML output, selected by setting
\texttt{htmlize-output-type}: \texttt{css}, \texttt{inline-css} (optimized for code
snippets), and \texttt{font} (simpler output, doesn't rely on CSS).  See
\href{http://htmlpreview.github.io/?https://github.com/hniksic/emacs-htmlize/blob/master/htmlize.el.html}{\texttt{htmlize.el.html}} for an example of generated HTML.

You can also use htmlize from your Emacs Lisp code.  When called
non-interactively, \texttt{htmlize-buffer} and \texttt{htmlize-region} will
return the resulting HTML buffer, but will not change current
buffer or move the point.  htmlize will do its best to work on
non-windowing Emacs sessions but the result will be limited to
colors supported by the terminal.

htmlize aims for compatibility with Emacsen version 22 and later.
Please let me know if it doesn't work on the version of XEmacs or
GNU Emacs that you are using.  The package relies on the presence
of CL extensions, especially for cross-emacs compatibility.

\end{document}
